% Options for packages loaded elsewhere
\PassOptionsToPackage{unicode}{hyperref}
\PassOptionsToPackage{hyphens}{url}
%
\documentclass[
  ignorenonframetext,
]{beamer}
\usepackage{pgfpages}
\setbeamertemplate{caption}[numbered]
\setbeamertemplate{caption label separator}{: }
\setbeamercolor{caption name}{fg=normal text.fg}
\beamertemplatenavigationsymbolsempty
% Prevent slide breaks in the middle of a paragraph
\widowpenalties 1 10000
\raggedbottom
\setbeamertemplate{part page}{
  \centering
  \begin{beamercolorbox}[sep=16pt,center]{part title}
    \usebeamerfont{part title}\insertpart\par
  \end{beamercolorbox}
}
\setbeamertemplate{section page}{
  \centering
  \begin{beamercolorbox}[sep=12pt,center]{part title}
    \usebeamerfont{section title}\insertsection\par
  \end{beamercolorbox}
}
\setbeamertemplate{subsection page}{
  \centering
  \begin{beamercolorbox}[sep=8pt,center]{part title}
    \usebeamerfont{subsection title}\insertsubsection\par
  \end{beamercolorbox}
}
\AtBeginPart{
  \frame{\partpage}
}
\AtBeginSection{
  \ifbibliography
  \else
    \frame{\sectionpage}
  \fi
}
\AtBeginSubsection{
  \frame{\subsectionpage}
}
\usepackage{amsmath,amssymb}
\usepackage{iftex}
\ifPDFTeX
  \usepackage[T1]{fontenc}
  \usepackage[utf8]{inputenc}
  \usepackage{textcomp} % provide euro and other symbols
\else % if luatex or xetex
  \usepackage{unicode-math} % this also loads fontspec
  \defaultfontfeatures{Scale=MatchLowercase}
  \defaultfontfeatures[\rmfamily]{Ligatures=TeX,Scale=1}
\fi
\usepackage{lmodern}
\usetheme[]{CambridgeUS}
\ifPDFTeX\else
  % xetex/luatex font selection
\fi
% Use upquote if available, for straight quotes in verbatim environments
\IfFileExists{upquote.sty}{\usepackage{upquote}}{}
\IfFileExists{microtype.sty}{% use microtype if available
  \usepackage[]{microtype}
  \UseMicrotypeSet[protrusion]{basicmath} % disable protrusion for tt fonts
}{}
\makeatletter
\@ifundefined{KOMAClassName}{% if non-KOMA class
  \IfFileExists{parskip.sty}{%
    \usepackage{parskip}
  }{% else
    \setlength{\parindent}{0pt}
    \setlength{\parskip}{6pt plus 2pt minus 1pt}}
}{% if KOMA class
  \KOMAoptions{parskip=half}}
\makeatother
\usepackage{xcolor}
\newif\ifbibliography
\usepackage{longtable,booktabs,array}
\usepackage{calc} % for calculating minipage widths
\usepackage{caption}
% Make caption package work with longtable
\makeatletter
\def\fnum@table{\tablename~\thetable}
\makeatother
\usepackage{graphicx}
\makeatletter
\def\maxwidth{\ifdim\Gin@nat@width>\linewidth\linewidth\else\Gin@nat@width\fi}
\def\maxheight{\ifdim\Gin@nat@height>\textheight\textheight\else\Gin@nat@height\fi}
\makeatother
% Scale images if necessary, so that they will not overflow the page
% margins by default, and it is still possible to overwrite the defaults
% using explicit options in \includegraphics[width, height, ...]{}
\setkeys{Gin}{width=\maxwidth,height=\maxheight,keepaspectratio}
% Set default figure placement to htbp
\makeatletter
\def\fps@figure{htbp}
\makeatother
\usepackage{soul}
\setlength{\emergencystretch}{3em} % prevent overfull lines
\providecommand{\tightlist}{%
  \setlength{\itemsep}{0pt}\setlength{\parskip}{0pt}}
\setcounter{secnumdepth}{-\maxdimen} % remove section numbering
\ifLuaTeX
  \usepackage{selnolig}  % disable illegal ligatures
\fi
\IfFileExists{bookmark.sty}{\usepackage{bookmark}}{\usepackage{hyperref}}
\IfFileExists{xurl.sty}{\usepackage{xurl}}{} % add URL line breaks if available
\urlstyle{same}
\hypersetup{
  pdftitle={EDF 5484 Educational Data Analysis},
  pdfauthor={Russell Almond},
  hidelinks,
  pdfcreator={LaTeX via pandoc}}

\title{EDF 5484 Educational Data Analysis}
\author{Russell Almond}
\date{}

\begin{document}
\frame{\titlepage}

\begin{frame}{EDF 5484 Educational Data Analysis}
\protect\hypertarget{edf-5484-educational-data-analysis}{}
Introduction and Welcome
\end{frame}

\begin{frame}{Who am I?}
\protect\hypertarget{who-am-i}{}
\includegraphics[width=\textwidth,height=0.75\textheight]{img/sometimes-you-need-an-old-white-man-with-a-beard.jpg}
Russell Almond
\end{frame}

\begin{frame}{Contact info}
\protect\hypertarget{contact-info}{}
\begin{itemize}
\tightlist
\item
  \url{http://ralmond.net/}
\item
  Email: \href{mailto:ralmond@fsu.edu}{\nolinkurl{ralmond@fsu.edu}}
\item
  Office Hours, By appointment (appointment link:
  \url{https://doodle.com/mm/russellalmond/book-a-time})
\item
  \url{https://fsu.zoom.us/my/ralmond}
\item
  On Campus, TTH Afternoons
\item
  STB 3204-J
\item
  Has ADHD
\end{itemize}

Questions welcome during lectures (speak up if on Zoom)
\end{frame}

\begin{frame}{Office Hours}
\protect\hypertarget{office-hours}{}
\begin{itemize}
\tightlist
\item
  I'm replacing the formal ``office hours'' with a ``coffee hour'' and
  ``tea time'' after my classes.

  \begin{itemize}
  \tightlist
  \item
    Coffee Hour: T,Th 11:30 in my Zoom Room
    \url{https://fsu.zoom.us/my/ralmond})
  \item
    Tea Time: W,F in my office 3204-J
  \end{itemize}
\item
  Feel free to drop by with technical questions, or just to chat
\item
  Zoom and office door open (although I may be in kitchen or 3204 common
  area).
\item
  Appointment link:
  \url{https://doodle.com/mm/russellalmond/book-a-time}
\end{itemize}
\end{frame}

\begin{frame}{Course Text}
\protect\hypertarget{course-text}{}
\textbf{Required:}

\begin{itemize}
\item
  Gelman, A., Hill, J. \& Vehtari, A. (2022) \emph{Regression and Other
  Stories}. Cambridge University Press.
\item
  Vehtari's course page: \url{https://avehtari.github.io/ROS-Examples/}

  \begin{itemize}
  \tightlist
  \item
    R code for book examples
  \item
    Data for homework problems.
  \item
    PDF of book
  \end{itemize}
\end{itemize}

\textbf{Additional Readings:}

\begin{itemize}
\tightlist
\item
  Andy Gelman's Blog: \url{http://andrewgelman.com/}
\end{itemize}
\end{frame}

\begin{frame}{Posit Cloud}
\protect\hypertarget{posit-cloud}{}
\end{frame}

\begin{frame}{Software}
\protect\hypertarget{software}{}
Will do most work through Posit Studio:

\url{https://posit.cloud/spaces/323676/join?access_code=-cWB4jXG0_PEIPItmrQWUzfsYsW4GZhTToU-Ork0}

Free Student Account.
\end{frame}

\begin{frame}{Concepts Covered (1)}
\protect\hypertarget{concepts-covered-1}{}
\begin{itemize}
\item
  Identify situations in which a particular generalized linear model is
  appropriate to answer a given research question with a given data set.
\item
  Perform common data analyses operations with generalized linear models
  in the R programming language.
\item
  Interpret the results of generalized linear models in terms of the
  original research questions
\item
  Construct graphical displays (figures and tables) which summarize the
  results of a data analysis.
\end{itemize}
\end{frame}

\begin{frame}{Concepts Covered (2)}
\protect\hypertarget{concepts-covered-2}{}
\begin{itemize}
\item
  Identify situations in which transforming the variables in the data
  set before analysis is appropriate.
\item
  Interpret the results of analyses with transformed variables.
\item
  Employ common diagnostic statistics and plots to identify and correct
  difficulties with (generalized) linear models.
\item
  Simulate data from a (generalized) linear model.
\item
  Produce statistical graphics to explain key results of generalized
  linear models.
\item
  Write up the results of statistical analyses in a form suitable for
  the results section of a scientific journal article.
\end{itemize}
\end{frame}

\begin{frame}{Flipping The Class}
\protect\hypertarget{flipping-the-class}{}
\begin{itemize}
\item
  With a few exception, I will not lecture
\item
  Lecture notes posted in the class are for reference: I may or may not
  spontaneously talk about them.
\item
  Before each class, one of the chapters of Gelman, Hill \& Vehtari will
  be assigned for reading.
\item
  Students will take turns working the homework problems in class.
  (\emph{Case Studies})
\end{itemize}
\end{frame}

\begin{frame}{Evidence Mix}
\protect\hypertarget{evidence-mix}{}
\begin{itemize}
\item
  Case Studies (1/2) -- Need to be able to get the discussion going, not
  necessarily finish the problem.
\item
  Midterm Project (1/4)
\item
  Final Project (1/4)
\end{itemize}

\emph{Evidence turned in after May 1st will not be considered unless
there are extenuating circumstances (i.e., you are getting an
incomplete)}
\end{frame}

\begin{frame}{Case Studies (1/2)}
\protect\hypertarget{case-studies-12}{}
\begin{itemize}
\tightlist
\item
  Serves dual purpose:

  \begin{itemize}
  \tightlist
  \item
    Forces you to verbalize your ideas about statistics, this promotes
    readiness-to-learn
  \item
    Provides me with formative feedback
  \end{itemize}
\item
  NOT graded on correctness
\item
  You are assigned problem (from book or Posit Cloud).
\item
  Prepare solution

  \begin{itemize}
  \tightlist
  \item
    or partial solution if you get stuck
  \end{itemize}
\item
  Present as much as you completed in class

  \begin{itemize}
  \tightlist
  \item
    Class will help you complete
  \end{itemize}
\item
  Currently 8 are scheduled, but could be more/less depending on
  schedule.
\end{itemize}
\end{frame}

\begin{frame}{Case Study Procedures}
\protect\hypertarget{case-study-procedures}{}
\begin{itemize}
\tightlist
\item
  There will be a Canvas discussion forum for posting Case Study
  solutions (so you can access them easily from class).
\item
  Must be present on day of completion
\item
  We will try to scheduled approximately when they are due when assigned

  \begin{itemize}
  \tightlist
  \item
    Let me know about planned travel, other commitments.
  \item
    In case of excused absence will try to reschedule
  \end{itemize}
\end{itemize}
\end{frame}

\begin{frame}{Midterm/Final Project (1/4 each)}
\protect\hypertarget{midtermfinal-project-14-each}{}
\begin{itemize}
\item
  Find a data set and a research question.
\item
  (about 6 weeks before final due date) Get data set and research
  question approved.
\item
  Analyze data (however you feel appropriate)
\item
  (2 weeks before final due date): Turn in draft for feedback
\item
  (1 week before final due date): Feedback returned
\end{itemize}

\emph{Final paper due.}
\end{frame}

\begin{frame}{Project Scoring Rubric}
\protect\hypertarget{project-scoring-rubric}{}
\begin{longtable}[]{@{}
  >{\raggedright\arraybackslash}p{(\columnwidth - 2\tabcolsep) * \real{0.8611}}
  >{\raggedright\arraybackslash}p{(\columnwidth - 2\tabcolsep) * \real{0.1389}}@{}}
\toprule\noalign{}
\begin{minipage}[b]{\linewidth}\raggedright
Criterion
\end{minipage} & \begin{minipage}[b]{\linewidth}\raggedright
Points
\end{minipage} \\
\midrule\noalign{}
\endhead
Research question properly stated and motivated & 10 \\
Data sufficiently described & 10 \\
Analysis Research question properly stated and motivated & 10 \\
Data sufficiently described & 10 \\
Analysis method appropriate for research question and data & 10 \\
Analysis correctly carried out & 10 \\
Graphics and tables included to support analysis & 10 \\
Diagnostic analyses carried out to check appropriateness of model &
10 \\
Results of analysis properly interpreted and summarized & 10 \\
Limitations of analysis recognized & 10 \\
Document text follows style/usage conventions for academic papers &
10 \\
\bottomrule\noalign{}
\end{longtable}
\end{frame}

\begin{frame}[fragile]{Style Points}
\protect\hypertarget{style-points}{}
\begin{itemize}
\item
  Commenting Code
\item
  Using descriptive variable and function names
\item
  Formatting code properly (indentation to show program structure, white
  space, using \texttt{\textless{}-} for assignments)
\item
  Using human readable label in plots and tables
\item
  Editing out dead-end code
\item
  Including information about data sources and meta-data.
\item
  Documenting function inputs, outputs and side-effects.
\item
  Properly parameterizing expressions (i.e., avoiding hard coded
  constants).
\end{itemize}
\end{frame}

\begin{frame}{Project Schedule}
\protect\hypertarget{project-schedule}{}
(On Canvas) Midterm paper due just before spring break. Final paper due
at the end of finals week.
\end{frame}

\begin{frame}{Collaborations}
\protect\hypertarget{collaborations}{}
\begin{itemize}
\tightlist
\item
  Case Studies

  \begin{itemize}
  \tightlist
  \item
    Larger case studies can be assigned as a team
  \item
    Use Blackboard discussion groups for help
  \end{itemize}
\item
  Projects

  \begin{itemize}
  \tightlist
  \item
    Collaboration is possible but must be approved in advance
  \item
    Each team member provides a unique piece
  \end{itemize}
\end{itemize}
\end{frame}

\begin{frame}{Plagiarism}
\protect\hypertarget{plagiarism}{}
\begin{itemize}
\tightlist
\item
  Plagiarism is a serious problem

  \begin{itemize}
  \tightlist
  \item
    People have LOST THEIR DEGREES
  \item
    First time violations will be reported
  \item
    See FSU policies and resources
  \item
    Turn-it-in used for projects
  \item
    Credit your sources!
  \end{itemize}
\end{itemize}
\end{frame}

\begin{frame}{FSU Teaching}
\protect\hypertarget{fsu-teaching}{}
\begin{itemize}
\tightlist
\item
  Don't use forums, FSU emails \&c for personal purposes

  \begin{itemize}
  \tightlist
  \item
    All FSU emails are public record
  \item
    \emph{Must} use FSU email for grade-related requests
  \end{itemize}
\item
  \emph{Don't use FSU Resources to download copyrighted material not
  related to your class work or research!}
\item
  Beware of leaving Bit Torrent clients on while connected to FSU
  network
\end{itemize}
\end{frame}

\begin{frame}{How's My Teaching?}
\protect\hypertarget{hows-my-teaching}{}
Dial: \st{644-5203}

Email: \href{mailto:ralmond@fsu.edu}{\nolinkurl{ralmond@fsu.edu}}

Let me know if you have problems reading material, especially if you
have trouble distinguishing red from black.
\end{frame}

\begin{frame}{Turning the \st{Battleship} Destroyer}
\protect\hypertarget{turning-the-battleship-destroyer}{}
👍 (Thumbs up) I get it already, you can skip ahead

👎 (Thumbs down) I'm confused, give me more detail

👏 (Hand up) Stop, I have a question (or issue)

What to do if you are lost

\begin{itemize}
\tightlist
\item
  Confusion is a part of learning

  \begin{itemize}
  \tightlist
  \item
    But it should be temporary
  \end{itemize}
\item
  Ask questions in class!
\item
  Come visit during office hours, or make an appointment
\item
  Be specific about what is confusing you
\end{itemize}
\end{frame}

\end{document}
